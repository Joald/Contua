\documentclass{article}
\usepackage[utf8]{inputenc}
\usepackage{amsfonts}
\usepackage{soul}
\usepackage{textcomp}
\usepackage{color}
\usepackage{enumitem}
\usepackage{amsmath}
\usepackage{amssymb}
\usepackage{mathtools}
\usepackage{listings}
\usepackage{amsthm} 
\usepackage[dvipsnames]{xcolor}
\usepackage[a4paper, total={6.5in, 10in}]{geometry}
\definecolor{dkgreen}{rgb}{0,0.6,0}
\definecolor{gray}{rgb}{0.5,0.5,0.5}
\definecolor{mauve}{rgb}{0.58,0,0.82}
\lstset{frame=tb,
    language=Haskell,
    aboveskip=3mm,
    belowskip=3mm,
    showstringspaces=false,
    columns=flexible,
    basicstyle={\small\ttfamily},
    numbers=none,
    numberstyle=\tiny\color{gray},
    keywordstyle=\color{blue},
    commentstyle=\color{dkgreen},
    stringstyle=\color{mauve},
    breaklines=true,
    breakatwhitespace=true,
    escapeinside={(*}{*)},          % if you want to add LaTeX within your code
    tabsize=4
}

\title{Overview of the Contua programming language.}
\author{Jacek Olczyk}
\date{Feb 2019}

\begin{document}
\maketitle
\section{Introduction}
\paragraph{Contua} is a strongly typed functional programming language. It is conceived on a basis of the idea that the language should force the programmer to write all the functions continuation-style. It is not immediately clear what measures needs to be taken to disallow regular functions. I needed to come up with some conventions to achieve that result.
\section{The syntax}
Since the language will be pretty complicated in the 'backend', I tried to keep the syntax as simple as possible. In this grammar I omitted the whitespace rules.
\begin{align*}
% program
\texttt{program}  =\ & \texttt{\{ \color{gray}typeDecl \color{black} \}, \{ \color{red}funDecl \color{black} \}}\\
% funDecl
\texttt{\color{red}funDecl\color{black}} =\ & \texttt{\color{ForestGreen}type\color{black}, '::', id, \{ id \}, "=", \color{RoyalPurple}expr\color{black}}\\
% typeDecl
\texttt{\color{gray}typeDecl\color{black}} =\ &\texttt{\color{blue}'type'\color{black}, ID, \{ id \}, "=", \color{RubineRed}typeCtor\color{black}, \{ "|", \color{RubineRed}typeCtor \color{black} \}}\\
% type
\texttt{\color{ForestGreen}type\color{black}} =\ &\texttt{\color{CadetBlue}basicType \color{black} | (\color{ForestGreen}type\color{black}, \{ '->', \color{ForestGreen}type \color{black} \}) | "(", \color{ForestGreen}type\color{black}, ")"}\\
% basicType
\texttt{\color{CadetBlue}basicType} =\ &	\texttt{\color{RubineRed}typeCtor \color{black}| id | "[", id, "]"}\\
% typeCtor
\texttt{\color{RubineRed}typeCtor\color{black}} =\ & \texttt{ID, \{ \color{ForestGreen}type \color{black} \}}\\
% function
% \texttt{\color{brown}function\color{black}} =\ & \texttt{}\\
% expr
\texttt{\color{RoyalPurple}expr\color{black}} =\ &  \texttt{
	% variables
	id
	% type ctors
	| ID
	% +
	| \color{RoyalPurple}expr\color{black}, "+", \color{RoyalPurple}expr \color{black} 
	% -
	| \color{RoyalPurple}expr\color{black}, "-", \color{RoyalPurple}expr \color{black} 
	% *
	| \color{RoyalPurple}expr\color{black}, "*", \color{RoyalPurple}expr \color{black}
	% - unary
	| "-", \color{RoyalPurple}expr \color{black}
|}\\
&\texttt{
	% ()
	| "(", \color{RoyalPurple}expr\color{black}, ")" 
	% application
	| \color{RoyalPurple}expr\color{black}, \color{RoyalPurple}expr \color{black} 
	% lambda
	| \color{blue}'fn'\color{black}, \{ id \}, ".", \color{RoyalPurple}expr \color{black} 
	% listExpr
	| \color{RawSienna}listExpr \color{black} 
|}\\
&\texttt{
	% where
	| \color{RoyalPurple}expr\color{black}, \color{blue}'where'\color{black}, \color{red} funDecl\color{black}, \{ \color{blue}'and'\color{black}, \color{red}funDecl \color{black} \} 
	% let in
	| \color{blue}'let'\color{black}, \color{red}funDecl\color{black}, \color{blue}'in'\color{black}, \color{RoyalPurple}expr \color{black} 
|}\\
&\texttt{
	% match
	| \color{blue}'match'\color{black}, \color{RoyalPurple}expr\color{black}, \color{blue}'with'\color{black}, \{ "|", \color{RoyalPurple}expr\color{black}, '=>', \color{RoyalPurple}expr \color{black} \}
|}\\
&\texttt{
	% if then else
	| \color{blue}'if'\color{black}, \color{Goldenrod}bexpr\color{black}, \color{blue}'then'\color{black}, \color{RoyalPurple}expr\color{black}, \color{blue}'else'\color{black}, \color{RoyalPurple}expr\color{black}
}\\
% bexpr
\texttt{\color{Goldenrod}bexpr\color{black}} =\ &\texttt{
	% boolean variable
	id
	% conjunction
	| \color{Goldenrod}bexpr\color{black}, \color{blue}'and'\color{black}, \color{Goldenrod}bexpr \color{black}
	% disjunction
	| \color{Goldenrod}bexpr\color{black}, \color{blue} 'or'\color{black}, \color{Goldenrod}bexpr \color{black}
	% negation
	| \color{blue} 'or'\color{black}, \color{Goldenrod}bexpr \color{black} 
	% less
	| \color{RoyalPurple}expr\color{black}, '<=', \color{RoyalPurple}expr\color{black}
}\\
% listExpr
\texttt{\color{RawSienna}listExpr\color{black}} =\ &\texttt{
	% list literal
	[\color{RoyalPurple}expr\color{black}, \{",", \color{RoyalPurple}expr \color{black}\}]
	% head/tail
	| \color{RoyalPurple} expr\color{black}, ":", \color{RawSienna}listExpr \color{black}
	% concatenation
	| \color{RawSienna}listExpr\color{black}, "++", \color{RawSienna}listExpr \color{black}
}
\end{align*}
\section{Forcing continuation style}
\end{document}